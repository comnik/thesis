\documentclass[../index.tex]{subfiles}

\begin{document}

As more and more industries adopt data-driven decision-making and
companies are increasingly looking to provide digital services, the
capability to extract actionable insights from many continuous data
streams in a timely manner is becoming a critical competitive
advantage.

From an end user perspective, any results derived from the data
flowing through an organization should above all be correct,
incorporate the most recent data available, and reflect a consistent
view of the organization as of some point in time. Additionally, it is
not acceptable any longer to hold information in mutable cells. Rather
information must be recorded in such a way, that historical states can
be recovered for analytical and auditing purposes.

To that end, many companies are starting to adapt event-driven
architectures, in which data producers and consumers are uncoupled via
a shared, append-only log of records. Services coordinate reactively,
via the arrival of new data. While this approach makes for
understandable, flexible systems, it puts a larger burden on data
consumers to construct and maintain consistent, up-to-date views on
the subset of information that is relevant to them.

Purpose-built stream processing systems can meet very high throughputs
and near real-time latencies, but lack support for expressive
programming models and strong consistency guarantees. Specialized
graph-, OLAP-, and time-series databases on the other hand support
much more complex algorithms, but their ad-hoc interaction model does
not fit well into reactive, near real-time environments.

Only recently have dataflow systems emerged, that support strongly
consistent, incremental maintenance of complex computations over
high-throughput data streams. The resulting systems respond
efficiently to unbounded, arbitrarily changing inputs, support
distributed execution out-of-the-box, and thus provide a foundation
onto which data consumers can off-load view maintenance. These systems
sit at the intersection of databases and stream-processing and thus
merit re-visiting established practices in both fields.

\end{document}

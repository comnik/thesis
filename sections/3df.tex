\documentclass[../index.tex]{subfiles}

\begin{document}

With a system like Differential Dataflow (\emph{Differential})
available it becomes feasible to rethink the boundaries between the
domain of analytical databases and that of near real-time stream
processors.

In previous work (\cite{declarative}) we therefore built 3DF (short
for \emph{Declarative Differential Dataflow}), a dynamic query engine
built on top of Differential. While Differential computations are
compiled into static binaries, 3DF allows clients to register and
unregister relational queries at runtime and to receive output changes
via the network.

The existing 3DF codebase will act as baseline and reference
implementation for later chapters. Thus, this chapter is dedicated to
describe the design and implementation of 3DF at the outset of this
work.

\subsection{Data Model} \label{3df-data-model}

3DF imposes strong opinions on data modeling. In particular, we adopt
a unified, \emph{attribute-oriented} data model. The fundamental unit
of modeling are facts, represented by entity, attribute, value
triples. Although conceptually we are modelling a single space of
facts as it evolves through logical time, we keep track of individual
attributes in their own Differential collections. Attribute
collections are thus binary relations. Throughout this work we use the
terms \emph{attribute} and \emph{base relation} interchangeably. We
use names starting with a colon to identify attributes
(e.g. \texttt{:person/name}, \texttt{:comment/parent-post}).

An attribute-oriented data model corresponds to a fully normalized
relational model, and is heavily inspired by graph databases, RDF, and
research on triple stores. 3DF also draws heavily from database
systems like Datomic (\cite{datomic}) and LogicBlox
(\cite{aref2015design}).

We believe that important semantics — such as schema, access and
privacy policies, or compaction intervals — must be configurable at
the level of an individual attribute, rather than in aggregate
(e.g. on a relation in a traditional database schema), to satisfy the
need for evolvable information systems in the real-world. Other
organizational aspects such as documentation and namespacing are also
most naturally expressed at this fine-grained level. An important
effect of attribute-oriented modeling is the ability to source
attributes from different external systems even though they might
belong together from a domain modeling perspective.

It is well established that data processing systems with data models
optimized for a specific domain can outperform general-purpose systems
by orders of magnitude. Today, users can (in addition to traditional
RDBMS) choose from column-stores designed for analytical queries,
databases for time-series, graphs, or geographical data, or full-text
search engines. We share the view of \cite{aref2015design} towards
specialized databases. Their use is warranted in scenarios where they
can provide at least an order of magnitude in efficiency gains. Below
that, the cost of developing, maintaining, and integrating a
specialized system dominates. We suspect that this threshold is even
higher for many industrial users with no in-house competence in
designing data-processing systems.

Another important characteristic of modern analytical tasks is the
prevalence of categorical features. In a relational setting, features
correspond to distinct unary relations. In a traditional, row-oriented
data model, modeling such data will lead to very wide, sparse
tuples. Attribute-oriented data models can represent such data much
more efficiently, and materialize only necessary attributes, at the
cost of frequent multi-way joins across relations of potentially
highly varying selectivities. These same insights drive the design of
modern, column-oriented analytical databases. We use the terms
\emph{attribute} and \emph{column} interchangeably throughout this
work.

We therefore think it justified to fix attribute-orientation as a
strong requirement for 3DF, and investigate instead whether novel
approaches in other areas (such as Differential Dataflow itself) allow
us to build systems that are both useful in real-world settings, and
still offer competitive performance across a wide class of use cases.

\subsection{Query Plan Language}

3DF itself does not enforce a specific query language, but is intended
to support relational query languages in general. We designed 3DF
specifically with the needs of a reactive Datalog engine in
mind. Datalog supports recursive rules and is thus well suited to
expressing subpattern queries on graphs, large n-way joins, and many
other common, relational computations in a concise way. The base
language can be easily extended into various domains (e.g. temporal
patterns, or probabilistic queries).

Datalog and other relational languages (such as SQL and its dialects)
relies on a small set of computational primitives: selections, joins,
projections, and set union. Datalog draws its expressive power from
recursive rules, which can be supported by iterating the query
evaluation to fixed point. With each new iteration, facts derived in
the previous evaluation are incorporated. The computation stops when
it reaches a fixed point, i.e. when no new facts are derived in an
iteration. On top of those primitives, we support stratified logical
negation (\emph{negation as set difference}).

Differential comes with a built-in two-way \texttt{join}
operator. Selections correspond to Differential's \texttt{filter}
operator. Set union is expressed as \texttt{concat}. Projections are
implemented as a \texttt{map} extracting the requested offsets from a
stream of tuples. Set difference is implemented via the
\texttt{antijoin} operator.

Timely and Differential support local iteration to fixed point,
meaning parts of the dataflow graph can be iterated independently from
one another and from the overall computation itself. Complex iteration
patterns, such as mutual recursion, can be expressed by means of
Differential's \texttt{Variable} abstraction. Variables represent
collections that are not completely defined until the entire dataflow
is constructed.

At the outset of this work, 3DF enforced set semantics over
Differential's multisets, via use of the \texttt{distinct}
operator. This changed over the course of this work and multiset
semantics are now reflected as is. The reasons for this are detailed
later in \autoref{set-vs-multiset}.

Within 3DF, query plans are represented by trees of operator
descriptions (\emph{plan stages}). In addition to plan stages for the
operators described above, we provide stages for various aggregations
on tuples, as well as data transforms and basic arithmetic via
built-in functions.

The leaf nodes of a query plan are made up out of \emph{data
  patterns}, which encode the supported access paths for facts:

\begin{itemize}
\item \textbf{MatchA := [?e a ?v]} reads an entire attribute
  collection $a$.
\item \textbf{MatchEA := [e a ?v]} reads all values associated with
  entity $e$ for attribute $a$.
\item \textbf{MatchAV := [?e a v]} reads all entity ids which have the
  value $v$ associated with them for attribute $a$.
\end{itemize}

This concludes our short summary of 3DF and its foundational
technologies. We've covered how 3DF provides a baseline of dynamic,
relational capabilities within an incremental, streaming setting. We
also gave arguments for 3DF's somewhat extreme stance on data model
flexibility and evolvability.

\end{document}

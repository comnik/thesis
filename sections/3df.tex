\documentclass[../index.tex]{subfiles}

\begin{document}

\subsection{Data Model}

We adopt a unified, attribute-oriented data model. The fundamental
unit of modeling are \textbf{facts}, which are represented by `[e a
  v]`-tuples (short for \textbf{entity}, \textbf{attribute},
\textbf{value}). Each triplet is to be read as "Entity e has
associated with it the attribute a, with a value v". 3DF currently
supports the following value types:

``` rust
enum Value \{
    \emph{/} An entity identifier
    Eid(u64),
    \emph{/} An attribute identifier
    Aid(String),
    \emph{/} A string
    String(String),
    \emph{/} A boolean
    Bool(bool),
    \emph{/} A 64 bit signed integer
    Number(i64),
    \emph{/} A 32 bit rational
    Rational32(Rational32),
    \emph{/} Milliseconds since midnight, January 1, 1970 UTC
    Instant(u64),
    \emph{/} A 16 byte unique identifier.
    Uuid([u8; 16]),
\}
```

We use 64bit unsigned integers to identify entities and string names
starting with a colon to identify attributes (e.g. `:person/name`,
`:comment/parent-post`). While the exact choice of types doesn't
matter, it is this unification which frees us from having to compile
dataflows individually.

This flexible approach of breaking schemas down into individual
attributes corresponds to a fully normalized relational model, and is
heavily inspired by RDF, research on triple stores, or database
systems like Datomic and LogicBlox. \texttt{@TODO sources: rdf, triple
stores, Datomic, LogicBlox}. It allows to configure important
semantics, such as data validation, access and privacy policies,
compaction, and indexing strategy at a much more fine-grained level
than a traditional database schema could accomodate. Other
organizational aspects such as documentation and namespacing are also
most naturally expressed at the level of the individual attribute.

It is well established that data processing systems with data models
optimized for a specific domain can outperform general-purpose systems
by one or more orders of magnitude. Today, users can (in addition to
traditional RDBMS) choose from column-stores designed for analytical
queries, databases for time-series, graphs, or geographical data,
search engines, and many more @TODO.

We strongly believe that the next generation of query processors must
take into account the needs of the emerging data-science community, as
those will make up most of our users in the future. Motivating this
further and investigating all of the broad implications this will have
on database design is beyond the scope of this work, but we will point
out two important, exemplary concerns.

\textbf{Heterogeneous Sources}

We can no longer assume that all inputs to a computation come from the
same source (e.g. a traditional RDBMS). This is especially the case
for systems like 3DF, which are not concerned with data storage at
all. Inputs come from external systems, are stored in distributed
filesystems, or are provided interactively by users. Inputs can have
vastly different characteristics w.r.t volume, rate, and data type.

The popularity of microservice-based architectures also imply, that
modern query processors must integrate data from diverse sources in
order to even arrive at a consistent, global view of any single domain
entity.

This new reality is elegantly captured by a column-oriented approach,
where sourcing information, data schema, and other semantics can be
configured separately for each individual attribute.

\textbf{Categorical Data}

An important computational characteristic of modern statistical
learning is the prevalence of categorical features, typically
represented via one-hot encodings. In a relational setting, one-hot
encodings correspond to distinct unary relations for each distinct
category a given record may fall into.

In a traditional, row-oriented data model, modeling such data will
lead to very wide, sparse tuples. Colum-oriented data models can
represent such data much more efficiently and materialize only
neccessary attributes, but incur frequent multi-way joins across
relations of potentially highly varying selectivities.

Motivated by these considerations and with further inspiration from
the LogicBlox and Datomic systems[0, 1], we model all data in sixth
normal form, meaning that every attribute is represented as a separate
relation. This comes with a number of trade-offs, whose detailed
investigation is not the focus of this work. In addition to the
treatment in Aref et al.\footnote{[Aref et al., Design and Implementation of the LogicBlox System](../sources/logicblox.pdf)}, we find that this model is well suited to
dealing with many heterogeneous sources and maps naturally to Datalog
expressions.

\subsubsection{Implementation}

Although conceptually we are modelling a single space of facts as it
evolves through logical time, we keep track of individual attributes
in their own Differential collections. Attribute collections hold `(e,
v)` pairs.

\subsubsection{One Size Fits All?}

We share the view of Aref et al.\footnotemark[1]{} towards specialized
databases. Their use is warranted in scenarios where they can provide
at least an order of magnitude in efficiency gains. Below that, the
cost of developing, maintaining, and integrating a specialized system
dominates. We suspect that this threshold is even higher for many
industrial users with no in-house competence in designing
data-processing systems.

A similar approach has been taken by the Datomic database\footnote{[Datomic](www.datomic.com)} and the
EVE project\footnote{[EVE](incidentalcomplexity.com)}. We witnessed the former's outsized, positive impact on
developer productivity and system evolvability first-hand.

We therefore think it worthwile to investigate whether novel
approaches allows us to make general-purpose query engines that can
offer competitive performance across a wider class of use cases.

\subsection{Query Plan Language}

3DF itself does not enforce a specific query language, but is intended
to support relational query languages in general. We designed 3DF
specifically with the needs of a reactive Datalog engine in
mind. Datalog supports recursive rules and is thus well suited to
expressing subpattern queries on graphs, large n-way joins, and many
other common, relational computations in a concise way. The base
language can be easily extended into various directions (e.g. temporal
patterns, or probabilistic queries). Languages such as SQL and its
various dialects (as used in modern data-processing systems) use the
same primitives.

Any relational language relies on a small set of computational
primitives: selections, joins, projections, and set union. As
mentioned above, Datalog draws its expressive power from recursive
rules, which can be supported via iteration to fixed point. On top of
those primitives, we want to support stratified logical negation
("negation as set difference").

Differential comes with a built-in two-way `join` operator. Selections
correspond to Differential's `filter` operator. Set union is expressed
as `concat`. Projections are implemented as a `map` extracting the
requested offsets from a stream of tuples. Set difference is
implemented via the `antijoin` operator.

Timely and Differential support local iteration to fixed-point,
meaning parts of the dataflow graph can be iterated independently from
one another, and from the overall computation itself. Complex
iteration patterns, such as mutual recursion, can be expressed by
means of Differential's `Variable` abstraction. Variables represent
collections that are not completely defined yet.

While relational languages usually provide set semantics, we will be
reflecting the multi-set semantics implemented by Differential
collections. Wherever necessary, the `distinct` operator can be used
to enforce set semantics on a collection.

Within 3DF query plans are represented as trees of operator
descriptions (\textbf{plan stages}). In addition to plan stages for the
operators described above, we provide stages for various aggregations
on tuples, as well as data transforms and basic arithmetic via
built-in functions.

The leaf nodes of a query plan are made up out of \textbf{data patterns},
which encode the supported access paths for facts:

\begin{itemize}
\item \textbf{MatchA := [?e a ?v]} reads an entire attribute collection `a`.
\item \textbf{MatchEA := [e a ?v]} reads all values associated with entity `e`
for attribute `a`.
\item \textbf{MatchAV := [?e a v]} reads all entity ids which have the value `v`
associated with them for attribute `a`.
\end{itemize}

We do not support variables in attribute position. Neither do we
support a pattern for checking the existence of an individual fact
(e.g. `MatchFact := [e a v]`).

\subsection{Input Sources}

An important effect of attribute-oriented modeling is the ability to
source attributes from different external systems, even though they
might belong together from a domain modeling perspective.

3DF allows for generalized source operators, which feed one or more
attribute indices. Source operators connect to external systems or
read from the filesystem and break external data models down into
their individual attributes. They are also responsible for entity id
and timestamp assignment.

Decisions made by sources can have a significant impact on overall
system performance. Sources may stall the entire system by failing to
downgrade their capabilities, cause uneven load patterns by issuing
too small or too large batches, or overload the progress tracking
subsystem by assigning unnecessarily many distinct timestamps. While
highly relevant, these considerations are beyond the core focus of
this work.

\end{document}

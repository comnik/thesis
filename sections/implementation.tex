\documentclass[../index.tex]{subfiles}

\begin{document}

\subsection{Indexing}

In order to make use of the worst-case optimal dataflow-join framework
described in section \ref{case-join-ordering} we could no longer
maintain attributes as just an arranged collection of \texttt{(e v)}
pairs. The implementation of the \texttt{PrefixExtender} trait
(described later in section \ref{impl-hector}) for attributes requires
two additional traces, one to keep track of the number of extensions
the attribute would propose for a given prefix and a trace indexed in
a way suitable to efficiently validate proposals made by other
extenders.

Additionally, attributes might be placed at a stage in the delta query
pipeline, at which prefixes already bind the value symbol. In that
case, reversed versions of all three of the above arrangements must be
at hand.

We therefore introduced a \texttt{CollectionIndex} structure which
holds all three arrangements for a given direction and a given
attribute. 3DF workers maintain separate mappings from attribute names
to their forward and reverse collection indices. This simplifies the
types and ownership involved. Collection indices implement a subset of
Differential's arrangement API: an \texttt{import} method for
importing all internal traces into a top-level scope, as well as an
\texttt{enter\_at} method to bring all imported internal arrangements
into a nested scope with an additional timestamp co-ordinate.

\begin{lstlisting}[language=Rust, style=colouredRust][h!]
struct CollectionIndex<K, V, T> {
    count_trace: TraceKeyHandle<K, T, isize>,
    propose_trace: TraceValHandle<K, V, T, isize>,
    validate_trace: TraceKeyHandle<(K, V), T, isize>,
}
\end{lstlisting}

These finer-grained index types map onto different use cases
throughout 3DF, and allows us to avoid creating new arrangements
during synthesis. For example, reading out an entire attribute is best
served by the \texttt{validate\_trace}, whereas when joining two
attributes using 3DFs existing \texttt{Join} stage, we can now re-use
the appropriate traces:

\begin{verbatim}
(join [?a ?b] [?a ?c]) => (forward propose, forward propose)
(join [?a ?b] [?c ?a]) => (forward propose, reverse propose)
(join [?a ?b] [?a ?b ?c]) => (forward validate, new arrangement)
...
\end{verbatim}

Although not client-configurable at the time of this writing, it makes
sense to skip the reverse indices for certain attributes. An example
would be one-way mappings used for string interning, or unary
attributes indicating categorical features.

\subsection{Lazy Synthesis} \label{lazy-synthesis}

In chapter \ref{case-eagerness} we have seen that demand-driven
synthesis of query plans is a pre-requisite to needlessly
materializing rules which materialize large numbers of tuples, but
would never even be used without other constraints of higher
selectivity present.

As explained in chapter \ref{3df}, 3DF used to synthesize all rules
eagerly. Over the course of this work, we extended 3DF workers with a
rule store. Upon receiving a \texttt{Register} request, workers now
merely store the provided plan (assuming it doesn't clash with a known
rule of the same name). Synthesis of a rule \texttt{q} will only
happen, once the first \texttt{Interest} command for \texttt{q} is
received.

A few more changes were required. In particular, a mechanism to gather
query dependencies had to be added. Queries can depend on attributes
and on other rules (that were registered in advance). This captured by
a simple struct:

\begin{lstlisting}[language=Rust, style=colouredRust][h!]
struct Dependencies {
    // NameExpr's used by this plan.
    names: HashSet<String>,
    // Attributes queries in Match* expressions.
    attributes: HashSet<Aid>,
}
\end{lstlisting}

Like all other aspects of synthesis, \texttt{Dependency} structs are
collected recursively starting at the leafs of the query plan (data
patterns such as \texttt{MatchA} and references via
\texttt{NameExpr}). The resulting set of rules requiring synthesis is
brought into a canonical order, to ensure that the same exact dataflow
graph is created on all workers. Redundant dependencies are removed.

At this point a decision must be made for each named dependency (for
which an existing arrangement is available), on whether to re-use the
arrangement or resolve the dependency into its bindings and synthesize
them from scratch. For the reasons explored in chapter
\ref{case-redundant-dataflows}, the default policy is to always
synthesize from scratch, re-using only attributes indices.

\subsection{Logging and Instrumentation} \label{logging}

A fair bit of instrumentation is necessary in order to monitor the
dataflow graph as it evolves over time, the number of tuples held in
arrangements, and many other metrics.

To that end, Timely Dataflow provides an extensible set of internal
dataflow streams, onto which internal events are published. Logged
events include among others the creation, shutdown, and scheduling of
operators, activity on communication channels, and progress tracking.

Differential Dataflow uses the same subsystem to provide additional
information about the creation and merging of trace batches. These are
sufficient to infer, for any relevant time, the number of arranged
tuples maintained at each arrangement in the dataflow.

Deriving relevant signals from the stream of logging events
dynamically via 3DF's reactive query engine seemed like a natural
fit. We created two new 3DF sources, \texttt{TimelyLogging} and
\texttt{DifferentialLogging}.

The \texttt{TimelyLogging} source allows clients to source only the
subset of logging attributes that they are interested in (c.f. listing
[@TODO below]). It also performs some pre-processing on the raw
logging streams, converting hierarchical scope addresses into the
corresponding edges of the dataflow graph.

\begin{lstlisting}[language=json][h!]
{:TimelyLogging
  {:attributes [:timely.event.operates/name
                :timely.event.operates/shutdown?]}}
\end{lstlisting}

Similarly, the \texttt{DifferentialLogging} source will not expose
batching and merging events directly, but rather derive from them
changes to the number of tuples held by each arrangement. Batch events
imply an increase in the number of tuples held (by the size of the
batch), merge events imply a decrease (the difference between the
compacted size of the merge result and the sum of the size of the
input batches).

Taken together, this makes it easy to write queries such as the
following, asking for the total number of tuples arranged at operators
that haven't been shut down.

\begin{verbatim}
;; total number of tuples arranged at alive operators

[:find (sum ?size)
 :where
 [?x :differential.event/size ?size]
 (not [?x :timely.event.operates/shutdown? true])]
\end{verbatim}

For making accurate measurements it also became neccessary to
side-step 3DFs default behaviour of forwarding results directly back
to clients, as doing so would introduce serialization and network
latency.

We therefore introduced a simple new type of plan stages called
\emph{sinks}, as well as a simple measurement sink called
\texttt{TheVoid}. This sink will swallow all its inputs and keep track
of the computational frontier. Whenever the frontier advances (thus
indicating the completion of a round of inputs), \texttt{TheVoid} will
append the number of milliseconds since the close of the previous
epoch to a logfile.

We then extended the \texttt{Interest} request to allow for an
optional sink configuration that, when provided, will cause 3DF to
forgo forwarding results via WebSocket and instead process them with
the specified sink.

Finally, we extended the \texttt{Interest} request to allow for an
optional granularity (in seconds), indicating the maximum rate at
which a client desires outputs. A granularity of one, for example,
would cause outputs to be delayed and consolidated up to the next full
second.

\subsection{Bindings} \label{bindings}

Over the course of this work it became evident that a few extensions
to the query plan representation would be necessary to (a) better
utilize the new primitive offered by worst-case optimal, n-way join
processing and (b) canonicalize plan representation in order to expose
more opportunities for sharing dataflows between multiple clients, as
described in chapter \ref{case-redundant-dataflows}.

From the discussion in \cite{veldhuizen2012leapfrog} we learn that the
Leapfrog Triejoin worst-case optimal join algorithm can be utilized to
implement many common features of relational query engines:
conjunctions (joins), disjunctions (unions), negations (antijoins),
and filtering by various predicates. This is achieved by additional
implementations of the core trie iterator interface. Similar
extensions exist within the worst-case optimal dataflow-join framework
due to \cite{ammar2018distributed}, which we have integrated into 3DF
as part of this work.

Any participant in a worst-case optimal dataflow join must implement
the \texttt{PrefixExtender} trait, which will be described in section
\ref{impl-hector}. For the purposes of the query language, we will
refer to such implementations as \emph{bindings}, for their property
of binding possible values to variables. We've provided
\texttt{PrefixExtender} implementations for attribute bindings,
constants, binary predicates, and negation.

Some bindings, such as those modeling predicates and negation, are
exclusively constraining (shrinking the space of possible values for a
given variable), others, such as attributes and constants, can also
provide values.

Converting between the existing clause language described in chapter
\ref{3df} and this new language of bindings is straightforward. The
basic data patterns translate as follows:

\begin{verbatim}
MatchA(?e, :a, ?v) => [attribute(?e, :a, ?v)]
MatchEA(e, :a, ?v) => [attribute(?e, :a, ?v), constant(?e, e)]
MatchAV(?e, :a, v) => [attribute(?e, :a, ?v), constant(?v, v)]
\end{verbatim}

The existing two-way join operator \texttt{Join(left, right)} is
resolved as the union of resolving both child plans into their
bindings.

@TODO antijoin

\texttt{Filter} plan stages map directly onto the corresponding
\texttt{binary\_predicate} binding.

Projections, aggregations, disjunctions, and functional transforms
remain unchanged.

\subsection{A Worst-Case Optimal N-Way Join Operator} \label{impl-hector}

In order to be able to express the worst-case optimal join strategy
explored throughout this work, a new query plan stage and
corresponding operator implementation had to be added. We've based the
present implementation off of \cite{dogsdogsdogs} and
\cite{dataflowjoin}. The resulting operator is called \emph{Hector}
and will be referred to as such throughout this chapter.

At a high-level, the Hector operator provides the following
capability: Given a set of bindings (as described in \ref{bindings})
and a set of target variables, find all possible variable assignments
that satisfy all bindings. The following Hector plan stage would
therefore be sufficient to express a simple triangle query:

\begin{verbatim}
{:Hector
 {:variables [?a ?b ?c]
  :bindings  [attribute(?a, :edge, ?b)
              attribute(?b, :edge, ?c)
              attribute(?a, :edge, ?c)]}}
\end{verbatim}

If no bindings are passed, Hector will throw an error. If a single
binding is passed (which must be an \texttt{attribute} binding, as no
other binding can provide tuples), Hector will merely perform a
projection onto the target variables.

If only two bindings are passed we have nothing to gain from the
worst-case optimal strategy, because for each of the source bindings
there will only be the one remaining binding left to propose
anything. Similarly, we do not benefit from delta queries here,
because a two-way join would not create any redundant intermediate
arrangements (as described in chapter \ref{case-join-state}).

For the general-arity case, Hector employs the delta query technique
explained in chapter \ref{case-join-state}. This means that a separate
dataflow will be constructed for each tuple-providing binding that may
experience change. Only \texttt{attribute} bindings can provide tuples
in our current implementation. By default, all attributes are assumed
to experience change. Continuing with our example, Hector would
therefore create three dataflows. For each of them, we will refer to
the generating binding as the \emph{source binding} and to the
corresponding dataflow as the \emph{delta pipeline}.

\begin{verbatim}
(1) d_edge(a, b) -> edge(b, c) -> edge(a, c)
(2) d_edge(b, c) -> edge(a, b) -> edge(a, c)
(3) d_edge(a, c) -> edge(a, b) -> edge(b, c)
\end{verbatim}

All delta pipelines are executed concurrently at the dataflow
level. This can lead to inconsistencies as we might derive the same
output change on multiple pipelines, when sources change
concurrently. Assume for example we have a graph containing the
triangle \texttt{[100 200 300]}, formed by the edges \texttt{(100
  200), (200 300), and (100 300)}. A single change \texttt{((100 200)
  -1)} to the \texttt{:edge} attribute will cause all three pipelines
to derive the same retraction \texttt{([100 200 300] -1)}.

To prevent this, we must impose a logical order on the computation. In
particular, we must ensure that the retraction of \texttt{@TODO}

Delta pipelines are therefore created within a new, nested scope
carrying the \texttt{AltNeu} timestamp type. Upon use, attributes such
as \texttt{:edge} are imported and wrapped with the corresponding
\texttt{AltNeu} timestamp. We cache imported arrangements by attribute
name, to prevent redundant imports.

Recall that a worst-case optimal join picks an appropriate variable
order (more on this crucial step later), along which a collection of
prefixes (initially containing only the empty prefix) is extended to
bind more and more symbols. While conceptually this is correct, it
does not translate directly into the dataflow setting. Dataflows must
always start with some source of input. In our case, the finest
grained source of input available are tuple-providing bindings,
i.e. the \texttt{attribute} binding — which already binds two symbols!

In order to consider changes to each individual variable separately,
we could break attributes further down into unified input collections
for each of their constituent variables. But this would have to be
done in a separate dataflow for each combination of bindings. Instead,
we would like to start with attribute inputs, and thus with prefixes
of length two.

In order to get away with this we must make sure to handle conflicts
on the variables of each source binding. Consider a Hector plan stage
involving (possibly amongst others) both an \texttt{attribute(?user,
  :user/name, ?name)} and a \texttt{constant(?name, "Alice")}
binding. When creating the delta pipeline starting from the
\texttt{:user/name} attribute, we would never give the constant
binding a chance to narrow down the collection of all usernames to
just those equal to "Alice".

It is straightforward to detect such conflicts for a given source
binding, as we can look for any of the remaining bindings for which
all of their variables are already bound by the prefix. In our
example, the constant binding binds only \texttt{?name}, which is
already bound by the prefix \texttt{[?user ?name]}.

The same can happen for attribute bindings. Consider
\texttt{attribute(?a, :edge, ?b)} and \texttt{attribute(?b, :edge,
  ?a)}, bindings which express a symmetry constraint between two nodes
in a directed graph. Sourcing the first attribute would lead to
\texttt{[?a ?b]} prefixes, and vice versa. In both cases, the other
binding would never get a chance to participate in prefix extension.

In our current implementation, Hector detects all conflicts, but only
handles those with constant bindings. This is done by filtering the
source binding accordingly.

For the following we will again assume that a suitable variable order
is at hand. We look at the variable order, and the variables bound by
the current prefix and determine from that the next variable $x$, to
which prefixes should be extended. Ignoring the source binding, we
then filter all other bindings down to only those that bind ("talk
about") $x$. Here we also skip bindings that are not \emph{ready} to
participate in prefix extension to $x$.

For example: an \texttt{attribute(?e, :a, ?v)} binding is \emph{ready}
to extend a prefix, which already binds \texttt{?e}, to \texttt{?v}
and vice versa. It is \emph{not ready} to extend a prefix such as
\texttt{[?a ?b ?c]} onto either \texttt{?e} or \texttt{?v}. In
contrast, a \texttt{constant(?c 123)} binding is \emph{always} ready
to extend any prefix to \texttt{?c}. Intuitively, a binding $B$ is
ready to participate in an extension of prefix $p$ to variable $x$,
iff $B$ binds $x$ and could do better than proposing \emph{all} of its
values for $x$. This is the same intuition behind the need to avoid
join orderings that would require computing the cartesian product of
two relations.

We have deferred the problem of choosing a suitable variable
ordering. The notion of \emph{readiness} allows us to define a
suitable ordering as one that includes all variables bound by any of
the participating bindings and which ensures that for each variable
$v$ there exists at least one binding ready to extend a prefix made up
out of the variables up to $v$, to $v$ itself. Therefore, given the
following bindings:

\begin{verbatim}
attribute(?a, :knows, ?b)
attribute(?b, :knows, ?c)
attribute(?c, :knows, ?d)
attribute(?a, :knows, ?d)
\end{verbatim}

\texttt{[?a ?b ?c ?d ?e]} would be a suitable order, whereas
\texttt{[?a ?c ?b ?d]} would'nt, because none of the bindings can
extend the prefix \texttt{[?a]} to \texttt{?c}.

Within the subset of \emph{suitable} orderings, there is the more
involved question of finding an \emph{optimal} ordering. Given the
bindings above and starting with the prefix \texttt{[?a ?b]}, it would
be equally valid to extend first to \texttt{?c} and then to
\texttt{?d} or vice versa.

Finding the optimal variable ordering is beyond the scope of this
work, but treated extensively in \cite{abo2016faq}.

\subsection{Multi-tenant Routing}

In chapter \ref{case-redundant-dataflows} we saw that multi-tenant
dataflows, where applicable, allow us to serve many clients without
creating additional dataflow elements. Our current implementation is
not fully automated, in that it requires co-operation from the
clients.

Multi-tenancy is achieved by tagging inputs with a token, uniquely
identifying the tenant it originated from. This way, the corresponding
results can be routed appropriately, such that any given tenant only
sees the results that are of interest to them. Our current
implementation re-uses the connection tokens assigned by the WebSocket
server. Client tokens are unsigned 64-bit integers.

Inputs are associated via parameter relations, where each tenant $t_i$
inputs their parameters $p_1, ..., p_n$ as pairs
$(t_i,p_1),...,(t_i,p_n)$. Assuming then a query such as the
following:

\begin{verbatim}
[:find ?tenant ?device ?speed
 :where
 [?device :device/speed ?speed]
 [?tenant :param/device ?device]]
\end{verbatim}

Tenants register their interest, indicating the offset at which tenant
tokens are to be found within the output tuples.

\begin{lstlisting}[language=json][h!]
{:Interest {:query q :tenant 0}}
\end{lstlisting}

Timely Dataflow makes use of \emph{exchange pacts} to define how data
is routed between operators. For single-tenant dataflows, all workers
therefore must be aware of which of them maintains the client
connection. For a given request, we call this worker the
\emph{owner}. Dataflow results are then exchanged using the following
pact, which simply routes all data to the owning worker:

\begin{lstlisting}[language=Rust, style=colouredRust][h!]
Exchange::new(move |_| owner as u64)
\end{lstlisting}

Each worker additionally maintains a mapping of query names to the
client tokens, which have expressed interest in receiving results from
that query. As long as all relevant results are exchanged correctly to
the owning worker, all interested clients will therefore receive
copies.

Knowledge about the owner is broadcasted during \emph{sequencing}, a
synchronizing dataflow into which all workers input all requests
received from clients. Sequencing ensures, that all workers process
the sequence of commands in the exact same order. Workers tag requests
that they have received from a client with their own worker id, before
broadcasting them via the sequencing dataflow.

Exchange logic becomes somewhat more involved in the multi-tenant
setting, because the routing decisions are now data dependent. Workers
must also maintain a mapping of tenants to their owning workers. New
entries into this mapping are created on \texttt{Interest} requests,
and removed upon \texttt{Disconnect} requests. Owning workers must
take care to broadcast appropriate \texttt{Disconnect} requests,
whenever any of their clients disconnects.

\begin{lstlisting}[language=Rust, style=colouredRust][h!]
Exchange::new(move |(tuple, t, diff)| {
    let tenant = tuple[tenant_offset];
    tenant_owner.borrow().get(&Token(tenant as usize))
})
\end{lstlisting}

\end{document}

\documentclass[../index.tex]{subfiles}

\begin{document}

Throughout this work we assume a setting in which clients arrive and
leave dynamically. Clients register relational queries and expect
result sets to be maintained in near real-time, as data from multiple
heterogeneous sources streams into the system.

We are therefore tackling the problem of multi-query optimization,
which is distinct from single-query optimization inasmuch as we do not
concern ourselves with producing optimal execution strategies for any
single query. Rather, we want to limit the number of ``disastrous''
plans picked, because those have an outsized negative effect on
\emph{all} users of the system. Our priorities are as follows:

\begin{enumerate}
  \item \textbf{Robustness} Above all, we would like for the resulting
    system to match user expecations, perform consistently well, and
    in a predictable manner.

  \item \textbf{Multi-user Support} In order to be useful as a
    general-purpose system, we must support many concurrent use cases
    and clients — be they analysts working with interactive sessions
    or distributed, long-running services processing large amounts of
    information.

  \item \textbf{Low-latency and High-Throughput} The attractiveness of
    stream processing derives from its performance characteristics,
    and the possibility to obtain results in a timely manner. While
    aiming for (1) and (2), we must take care to overall maintain
    these properties as much as possible.
\end{enumerate}

To this end, we will in \autoref{background} first provide some
background on the data-parallel dataflow model, modern stream
processors, and differential computation. \autoref{3df} describes 3DF,
a streaming relational query engine we've built previously, that will
serve as a baseline and reference implementation. 3DF enforces a
highly opinionated data model that fundamentally affects all other
considerations made throughout this work. We motivate these opinions
and provide a short overview of the state of 3DF before this work.

\autoref{known-techniques} is a survey of existing optimization
strategies from both the database and the stream processing
communities and a discussion of their applicability to our setting.

Across the subsequent chapters we then cover what we have found to be
the most significant obstacles that a system such as 3DF must overcome
to satisfy the above properties. These are (in no particular order):
the over-eager implementation of queries in a dynamic environment
(\autoref{case-eagerness}), the explosion of intermediate state
produced by high-arity joins (\autoref{case-join-state}), the robust
estimation of join cardinalities (\autoref{case-join-ordering}), and
navigating the trade-off between highly generalized plans whose
resources can be shared between many users and highly specialized
plans who can exploit additional optimization opportunities
(\autoref{case-redundant-dataflows}).

For each of these we describe the problems in general terms, provide
specific examples, propose strategies to remediate them, and evaluate
our implementation of these proposals. Implementation details on all
techniques discussed are provided later in \autoref{implementation}.

\autoref{case-skew} provides a similar analysis of the adverse effects
of non-uniformly distributed workloads and how certain aspects of
relational operators and 3DFs data model might exacerbate this
problem. While we do propose solutions, these are not implemented in
3DF at the time of this writing and.

We close with a discussion of less significant challenges discovered
over the course of this work, as well as of opportunities for further
work, in \autoref{future-work}, and summarize our contributions in
\autoref{conclusions}.

\end{document}

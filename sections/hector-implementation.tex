\documentclass[../index.tex]{subfiles}

\begin{document}

\subsection{Indexing}

Throughout this work we have often mentioned that base relations are
maintained in an indexed form that is shared across all dataflows. This 

In order to make use of the worst-case optimal dataflow-join framework
described in section \ref{case-join-ordering} we could no longer
maintain attributes as just an arranged collection of \texttt{(e v)}
pairs. The implementation of the \texttt{PrefixExtender} trait
(described later in section \ref{impl-hector}) for attributes requires
two additional traces, one to keep track of the number of extensions
the attribute would propose for a given prefix and a trace indexed in
a way suitable to efficiently validate proposals made by other
extenders.

Additionally, attributes might be placed at a stage in the delta query
pipeline, at which prefixes already bind the value symbol. In that
case, reversed versions of all three of the above arrangements must be
at hand.

We therefore introduced a \texttt{CollectionIndex} structure which
holds all three arrangements for a given direction and a given
attribute. 3DF workers maintain separate mappings from attribute names
to their forward and reverse collection indices. This simplifies the
types and ownership involved. Collection indices implement a subset of
Differential's arrangement API: an \texttt{import} method for
importing all internal traces into a top-level scope, as well as an
\texttt{enter\_at} method to bring all imported internal arrangements
into a nested scope with an additional timestamp co-ordinate.

\begin{lstlisting}[language=Rust, style=colouredRust][h!]
struct CollectionIndex<K, V, T> {
    count_trace: TraceKeyHandle<K, T, isize>,
    propose_trace: TraceValHandle<K, V, T, isize>,
    validate_trace: TraceKeyHandle<(K, V), T, isize>,
}
\end{lstlisting}

These finer-grained index types map onto different use cases
throughout 3DF, and allows us to avoid creating new arrangements
during synthesis. For example, reading out an entire attribute is best
served by the \texttt{validate\_trace}, whereas when joining two
attributes using 3DFs existing \texttt{Join} stage, we can now re-use
the appropriate traces:

\begin{verbatim}
(join [?a ?b] [?a ?c]) => (forward propose, forward propose)
(join [?a ?b] [?c ?a]) => (forward propose, reverse propose)
(join [?a ?b] [?a ?b ?c]) => (forward validate, new arrangement)
...
\end{verbatim}

Although not client-configurable at the time of this writing, it makes
sense to skip the reverse indices for certain attributes. An example
would be one-way mappings used for string interning, or unary
attributes indicating categorical features.

\subsection{Bindings} \label{bindings}

Over the course of this work it became evident that a few extensions
to the query plan representation would be necessary to (a) better
utilize the new primitive offered by worst-case optimal, n-way join
processing and (b) canonicalize plan representation in order to expose
more opportunities for sharing dataflows between multiple clients, as
described in chapter \ref{case-redundant-dataflows}.

From the discussion in \cite{veldhuizen2012leapfrog} we learn that the
Leapfrog Triejoin worst-case optimal join algorithm can be utilized to
implement many common features of relational query engines:
conjunctions (joins), disjunctions (unions), negations (antijoins),
and filtering by various predicates. This is achieved by additional
implementations of the core trie iterator interface. Similar
extensions exist within the worst-case optimal dataflow-join framework
due to \cite{ammar2018distributed}, which we have integrated into 3DF
as part of this work.

Any participant in a worst-case optimal dataflow join must implement
the \texttt{PrefixExtender} trait, which will be described in section
\ref{impl-hector}. For the purposes of the query language, we will
refer to such implementations as \emph{bindings}, for their property
of binding possible values to variables. We've provided
\texttt{PrefixExtender} implementations for attribute bindings,
constants, binary predicates, and negation.

Some bindings, such as those modeling predicates and negation, are
exclusively constraining (shrinking the space of possible values for a
given variable), others, such as attributes and constants, can also
provide values.

Converting between the existing clause language described in chapter
\ref{3df} and this new language of bindings is straightforward. The
basic data patterns translate as follows:

\begin{verbatim}
MatchA(?e, :a, ?v) => [attribute(?e, :a, ?v)]
MatchEA(e, :a, ?v) => [attribute(?e, :a, ?v), constant(?e, e)]
MatchAV(?e, :a, v) => [attribute(?e, :a, ?v), constant(?v, v)]
\end{verbatim}

The existing two-way join operator \texttt{Join(left, right)} is
resolved as the union of resolving both child plans into their
bindings.

@TODO antijoin

\texttt{Filter} plan stages map directly onto the corresponding
\texttt{binary\_predicate} binding.

Projections, aggregations, disjunctions, and functional transforms
remain unchanged.

\subsection{A Worst-Case Optimal N-Way Join Operator} \label{impl-hector}

In order to be able to express the worst-case optimal join strategy
explored throughout this work, a new query plan stage and
corresponding operator implementation had to be added. We've based the
present implementation off of \cite{dogsdogsdogs} and
\cite{dataflowjoin}, which itself is an implementation of the
\emph{Delta-BiGJoin} algorithm from \cite{ammar2018distributed}. The
resulting operator is called \emph{Hector} and will be referred to as
such throughout this chapter.

At a high-level, the Hector operator provides the following
capability: Given a set of bindings (as described in \ref{bindings})
and a set of target variables, find all possible variable assignments
that satisfy all bindings. The following Hector plan stage would
therefore be sufficient to express a simple triangle query:

\begin{verbatim}
{:Hector
 {:variables [?a ?b ?c]
  :bindings  [attribute(?a, :edge, ?b)
              attribute(?b, :edge, ?c)
              attribute(?a, :edge, ?c)]}}
\end{verbatim}

If no bindings are passed, Hector will throw an error. If a single
binding is passed (which must be an \texttt{attribute} binding, as no
other binding can provide tuples), Hector will merely perform a
projection onto the target variables.

If only two bindings are passed we have nothing to gain from the
worst-case optimal strategy, because for each of the source bindings
there will only be the one remaining binding left to propose
anything. Similarly, we do not benefit from delta queries here,
because a two-way join would not create any redundant intermediate
arrangements (as described in chapter \ref{case-join-state}).

For the general-arity case, Hector employs the delta query technique
explained in chapter \ref{case-join-state}. This means that a separate
dataflow will be constructed for each tuple-providing binding that may
experience change. Only \texttt{attribute} bindings can provide tuples
in our current implementation. By default, all attributes are assumed
to experience change. Continuing with our example, Hector would
therefore create three dataflows. For each of them, we will refer to
the generating binding as the \emph{source binding} and to the
corresponding dataflow as the \emph{delta pipeline}.

\begin{verbatim}
(1) d_edge(a, b) -> edge(b, c) -> edge(a, c)
(2) d_edge(b, c) -> edge(a, b) -> edge(a, c)
(3) d_edge(a, c) -> edge(a, b) -> edge(b, c)
\end{verbatim}

A worst-case optimal join picks an appropriate variable order (more on
this crucial step later), along which a collection of prefixes
(initially containing only the empty prefix) is extended to bind more
and more symbols. While conceptually this is correct, it does not
translate directly into the dataflow setting. Dataflows must always
start with some source of input. In our case, the finest grained
source of input available are tuple-providing bindings, i.e. the
\texttt{attribute} binding — which already binds two symbols!

In order to consider changes to each individual variable separately,
we could break attributes further down into unified input collections
for each of their constituent variables. But this would have to be
done in a separate dataflow for each combination of bindings. Instead,
we would like to start with attribute inputs, and thus with prefixes
of length two.

In order to get away with this we must make sure to handle conflicts
on the variables of each source binding. Consider a Hector plan stage
involving (possibly amongst others) both an \texttt{attribute(?user,
  :user/name, ?name)} and a \texttt{constant(?name, "Alice")}
binding. When creating the delta pipeline starting from the
\texttt{:user/name} attribute, we would never give the constant
binding a chance to narrow down the collection of all usernames to
just those equal to "Alice".

It is straightforward to detect such conflicts for a given source
binding, as we can look for any of the remaining bindings for which
all of their variables are already bound by the prefix. In our
example, the constant binding binds only \texttt{?name}, which is
already bound by the prefix \texttt{[?user ?name]}.

The same can happen for attribute bindings. Consider
\texttt{attribute(?a, :edge, ?b)} and \texttt{attribute(?b, :edge,
  ?a)}, bindings which express a symmetry constraint between two nodes
in a directed graph. Sourcing the first attribute would lead to
\texttt{[?a ?b]} prefixes, and vice versa. In both cases, the other
binding would never get a chance to participate in prefix extension.

In our current implementation, Hector detects all conflicts, but only
handles those with constant bindings. This is done by filtering the
source binding accordingly.

For the following we will again assume that a suitable variable order
is at hand. We look at the variable order, and the variables bound by
the current prefix and determine from that the next variable $x$, to
which prefixes should be extended. Ignoring the source binding, we
then filter all other bindings down to only those that bind ("talk
about") $x$. Here we also skip bindings that are not \emph{ready} to
participate in prefix extension to $x$.

For example: an \texttt{attribute(?e, :a, ?v)} binding is \emph{ready}
to extend a prefix, which already binds \texttt{?e}, to \texttt{?v}
and vice versa. It is \emph{not ready} to extend a prefix such as
\texttt{[?a ?b ?c]} onto either \texttt{?e} or \texttt{?v}. In
contrast, a \texttt{constant(?c 123)} binding is \emph{always} ready
to extend any prefix to \texttt{?c}. Intuitively, a binding $B$ is
ready to participate in an extension of prefix $p$ to variable $x$,
iff $B$ binds $x$ and could do better than proposing \emph{all} of its
values for $x$. This is the same intuition behind the need to avoid
join orderings that would require computing the cartesian product of
two relations.

From the set of relevant, ready bindings we derive a set of prefix
extenders. Each binding type corresponds to a prefix extender
implementation. Our extender implementations are detailed in the next
subsection. Extenders participate in a three-step process:

\textbf{Count} In the count step we determine for each source prefix
the extender that will propose the fewest extensions for it. This is
done by passing a stream of \texttt{(prefix count extender\_idx)}
through each extender's \texttt{count} operator. Initially all counts
are set to infinity. The count step results in a stream of
\emph{nominations} for each extender (prefixes for which the
respective extender has ``won the bid'').

\textbf{Propose} Nomination streams are passed through their
respective extender's \texttt{propose} operator. Propose
implementations materialize the extensions they counted in the
previous step and output a stream of \texttt{(prefix extension)}
pairs.

\textbf{Validate} Each extender's proposals are validated by all other
extenders. This is done by passing the stream of proposed pairs
through each extender's \texttt{validate} operator. Intuitively,
validate implementations compute the intersection of the set of
proposals with extensions that they themselves would have proposed,
had they been nominated to extend.

The output streams from the validation stage are concatenated together
and will form the stream of input prefixes for the next extension
stage. After the final stage, an additional projection onto the
variables requested by the user is performed.

\subsubsection{Prefix Extender Implementations}

Prefix extenders must provide \texttt{count}, \texttt{propose}, and
\texttt{validate} operators. While conceptually simple, these
implementations must not break the worst-case optimality, meaning that
they can only do work proportional to the lowest number of extensions
proposed by any of them (for a given extension stage).

The count operator must therefore avoid materializing and counting all
proposals, as doing so for every extender would immediately violate
the worst-case optimality. Implementations like Leapfrog TrieJoin and
our own will accept count implementations that take logarithmic time
(e.g. looking up prefixes in a suitable index).

As long as only the winning extender is asked to propose, the propose
implementation must only make sure to propose tuples with constant
delay.

A naive validation implementation might materialize all of its
proposals and compute the set intersection. This of course violates
worst-case optimality. However we may again take logarithmic time to
lookup proposed extensions in an index.

\textbf{CollectionExtender} \cite{dogsdogsdogs} gives an
implementation of a prefix extender for Differential collections of
key-value pairs, which we have largely adapted as is to implement
prefix extension for 3DF attributes. The resulting implementation
maintains attributes across three separate indices (count, propose,
and validate) backed by Differential arrangements. Arrangements
support lookups in logarithmic time.

\textbf{ConstantExtender} Prefix extension for constant bindings is
not backed by data other than the constant value. The count
implementation simply issues a count of one for each prefix. Propose
proposes for each prefix the constant value. Finally validate filters
the stream of proposed extensions to only include those which match
the constant value.

\textbf{BinaryPredicateExtender} Prefix extension for binary predicate
bindings is again not backed by data. Predicate extenders can never be
asked to propose, because it would have to propose infinitely many
extensions. The count implementation is therefore a noop, whereas the
propose implementation causes a runtime error, should the predicate
extender receive a nomination. The validate implementation will apply
the desired predicate to each proposed extension and to the prefix
offset binding the second argument.

\textbf{AntijoinExtender} Antijoin extenders implement negation as set
difference and thus wrap any other extender. Due to the stratification
requirement, antijoin extenders must not be asked to propose, and thus
implement count as a noop and propose as a runtime error. During
validation, antijoin extenders will subtract all extensions that the
wrapped extender has validated from the result stream.

\subsubsection{Suitable Variable Orderings}

We have deferred the problem of choosing a suitable variable
ordering. The notion of \emph{readiness} allows us to define a
suitable ordering as one that includes all variables bound by any of
the participating bindings and which ensures that for each variable
$v$ there exists at least one binding ready to extend a prefix made up
out of the variables up to $v$, to $v$ itself. Therefore, given the
following bindings:

\begin{verbatim}
attribute(?a, :knows, ?b)
attribute(?b, :knows, ?c)
attribute(?c, :knows, ?d)
attribute(?a, :knows, ?d)
\end{verbatim}

\texttt{[?a ?b ?c ?d ?e]} would be a suitable order, whereas
\texttt{[?a ?c ?b ?d]} would'nt, because none of the bindings can
extend the prefix \texttt{[?a]} to \texttt{?c}.

Within the subset of \emph{suitable} orderings, there is the more
involved question of finding an \emph{optimal} ordering. Given the
bindings above and starting with the prefix \texttt{[?a ?b]}, it would
be equally valid to extend first to \texttt{?c} and then to
\texttt{?d} or vice versa.

Finding the optimal variable ordering is beyond the scope of this
work, but treated extensively in \cite{abo2016faq}.

\subsubsection{Lexicographic Timestamps} \label{impl-altneu}

All delta pipelines are executed concurrently at the dataflow
level. This can lead to inconsistencies as we might derive the same
output change on multiple pipelines, when sources change
concurrently.

Assume for example we have a graph, initially containing only a single
edge $(100,300)$, and two delta pipelines:

\begin{verbatim}
(I) d(a,b) -> (b,c) -> (a,c)
(II) d(b,c) -> (a,b) -> (a,c)
\end{verbatim}

At some logical time $t$ we then input two new edges
$((100,200),t,+1)$ and $((200,300),t,+1)$. Crucially, both inputs are
then logically available in all relations at time $t$. Along pipeline
(I), the input edge $(100,200)$ will therefore match the indexed edge
$(200,300)$ and thus cause the derivation of a single output change
$((100,200,300),t,+1)$. Correspondingly, along pipeline (II) the input
edge $(200,300)$ will match the indexed edge $(100,200)$, leading to a
now redundant output change $((100,200,300),t,+1)$.

To prevent this, we must impose an arbitrary logical order on the
computation. In particular, we must ensure that input changes are not
logically available at all relations simultaneously (even though
physically, they might already be indexed). For this we fix an
arbitrary sequential order, e.g. $(a,b),(b,c),(a,c)$.

Delta pipelines are therefore created within a new, nested scope using
a modified, lexicographic timestamp type. Timestamps $t$ along each
pipeline are extended to $(t,a \in {0,1})$, where $a$ is a boolean
indicating the relative position of the source binding and the
extending relation. These extended timestamps are ordered
lexicographically, ensuring in particular that $(t,0) <
(t,1)$. Attribute arrangements are imported and wrapped with their
corresponding timestamps. We cache imported arrangements to prevent
redundant imports.

Revisiting the scenario from above, the input edges would now be
logically available in relation $(b,c)$ on pipeline (II), but not on
pipeline (I), because $(b,c)$ appears after $(a,b)$ in the relation
order. Accordingly, only pipeline (II) would now derive the correct
output change.

\end{document}

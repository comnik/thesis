\documentclass[../catalog.tex]{subfiles}

\begin{document}

\subsection{Problem}

When working with a declarative query language such as the one
provided by 3DF, users are encouraged to think of their domain in
terms of attributes, entities, and the relations that govern their
interaction. 3DF, borrowing from Datalog, provides the \emph{rule} —
fundamentally a named disjunction of conjunctions — as its core
abstraction. Rules may refer to each other or recursively to
themselves. Domain logic, constraints, and higher-level relationships
are all expressed as rules.

It is often the case, that users identify new rules that are sensible
abstractions within their domain, but when evaluated on their own,
materialize enormous amounts of tuples. Before this work, users of 3DF
would often run into this problem when modeling a problem
rule-by-rule, because 3DF would implement new rules immediately upon
registration. Additionally, in order to allow users to work off of
previously defined rules, 3DF would maintain rule results in a
Differential arrangement.

This is a prime example of violating the robustness property
identified in \ref{problem}: Users are merely following the
declarative interaction model (which we are actively encouraging), and
get punished for doing so!

\subsection{Example}

An example of this problematic scenario in the context of the social
graph domain is any transitive relation between persons, such as
\texttt{:person/knows} or \texttt{:person/friends-with}. A common rule
for recursively capturing such transitive closures might be:

\begin{verbatim}
[(knows ?a ?b) [?a :person/knows ?b]]
[(knows ?a b) [?a :person/knows ?x] (knows ?x ?b)]
\end{verbatim}

From the users perspective, this rule is tremendously useful, because
it abstracts away a complex (in the distributed setting), iterative
graph exploration problem for an arbitrary number of hops.

For 3DF on the other hand, immediately materializing this rule into a
Differential arrangement has disastrous consequences, as it will
materialize and index(!) the full transitive closure of the social
graph.

Another common instance of this problem arises when modeling access
control policies. The power of the query language allows for
straightforward expression of rich, property-based policies, or
simple, permission-based ones such as:

\begin{verbatim}
[(access? ?user ?entity) [?user :permission/read ?entity]]
\end{verbatim}

Eagerly deriving the \texttt{access?} relation would materialize all
read permissions for all users. Even more problematic, consider a rule
such as:

\begin{verbatim}
[(access? ?user ?entity) [?user :user/role "Admin"]]
\end{verbatim}

This rule makes sense from a domain-modeling point of view, but is not
even a valid rule in 3DF, because it doesn't bind the \texttt{?entity}
variable. Such a rule can only ever be used in combination with
additional ones.

\subsection{Remedy}

Luckily, generic rules are rarely used in isolation, i.e. without any
more specific constraints applied (in the same sense that a function
is rarely evaluated for all possible parameters, but rather for a much
smaller set of specific ones).

As to the example of \texttt{:person/knows}, an application might at
any given point be interested only in whether a few \emph{specific}
persons know of each other, or in exploring a limited number of hops
in the social network of a person. Therefore a first remedy might be
to hold back on synthesizing a rule until a client has expressed an
active interest in it.

Assume the definition of the \texttt{knows} rule is available on the
server, once a client has registered it, but the corresponding
dataflow isn't created until another client arrives with the following
query:

\begin{verbatim}
;; Is there a pair of users named Alice and Bob, 
;; which know of each other?

[:find ?x ?y
 :where
 [?x :person/firstname "Alice"]
 [?y :person/firstname "Bob"]
 (knows ?x ?y)]
\end{verbatim}

Now the resulting computation is much less problematic, because we
need only explore the transitive closure between persons named either
Alice or Bob. Deferring synthesis of \texttt{knows} has saved us a lot
of trouble in this case.

Additionally, by deferring evaluation, we have a complete picture of
user interest. This means that we don't have to maintain an arranged
version of the \texttt{knows} relation, but can immediately forward
tuples downstream.

A few significant changes were necessary to change 3DF's synthesis
approach from eager to lazy. These are detailed in
\ref{lazy-synthesis}.

\subsection{Conclusions}

As will be the topic of subsequent chapters, much deeper trade-offs
await when deciding what queries to evaluate as a unit and what parts
to maintain for re-use. For now, we merely point out that eager
materialization is problematic in that we lose the ability to even
reason about what the best evaluation strategy is. And we've seen that
semantically meaningful rules are often terribly underconstrained or
not derivable on their own.

Abstractions such as the \texttt{knows} rule increase the potential
for re-use of computations across many clients, but might dramatically
increase the magnitude of the result set.

On the other hand, as is established practice in query planning,
pushing more selective downstream operations back up into these
generic rules can greatly improve performance, at the cost of
increased \emph{specialization} of the dataflow — and thus reduce
potential opportunities for sharing it between clients. This problem
is compounded by the fact that a single bad decision, when
implementing a dataflow that many other flows depend on, means we are
stuck with the potentially horrific plan until all downstream
computations are shut down.

We will explore this trade-off in much greater detail in the following
scenarios.

\end{document}

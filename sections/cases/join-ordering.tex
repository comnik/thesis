\documentclass[../catalog.tex]{subfiles}

\begin{document}

\subsubsection{Problem}

As has been widely documented within the database community
(\cite{leis2015good}, \cite{lohman2014query}), many complex queries
are highly sensitive to join ordering. We have discussed the
established heuristical approaches originating from
\cite{selinger1979access} in \ref{known-techniques} and hinted at
their various shortcomings.

There we concluded that cardinality estimation under the traditional
assumptions (uniformity, uncorrelated join predicates, overlapping key
domains, and full access to meaningful statistics) can not avoid
disastrous plans in all cases.

While improvements in data modeling, on-line gathering of index
statistics, and dynamic execution approaches such as Eddies provide
many interesting avenues of inquiry, we are interested in approaches
that make disastrous plans impossible.

The reasons for this are two-fold: first, and most importantly,
disastrous query plans violate all three of the desired properties
outlined in \ref{problem}. Second, from experiences with 3DF itself,
and other declarative systems such as the Prolog language, we know
that the mere \emph{possibility} of unpredictable, severe performance
degradations break the fundamental promise of the declarative
abstraction, and force users to reason defensively about the system
runtime.

We also covered recent advances in the area of worst-case optimal join
processing (originally due to \cite{ngo2012worst} and independently
\cite{veldhuizen2012leapfrog}, more recently
\cite{ammar2018distributed} and \cite{ciucanu2015worst}) that
introduce n-way join algorithms with improved asymptotic complexity in
some cases.

\subsubsection{Remedy}

\end{document}

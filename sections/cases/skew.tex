\documentclass[../catalog.tex]{subfiles}

\begin{document}

The techniques discussed throughout the previous chapters, when taken
together, allow us to treat many-way relational joins as safe,
efficient operations in dynamic, multi-user environments. In the
context of 3DF specifically, they enable flexible data modeling that
can evolve with the needs of an organization and provide the
foundation on which the entire expressive range of a language like
Datalog is implemented. While continuously changing data distributions
posed a challenge to traditional methods of cardinality estimation,
they skewed datasets cause additional problems in a distributed
setting.

Skew problems arise whenever inputs are not distributed evenly across
workers. For stateless operators such as \texttt{map} or
\texttt{filter}, we are free to balance inputs evenly amongst
workers. This is generally not possible for stateful operators. In
particular, an operator such as \texttt{join} must be certain it will
see all inputs for a key $k$ on a specific worker, if it has seen any
input for $k$ on that worker. The physical distribution of data will
therefore tend to follow its logical distribution — which is rarely
uniform in practice.

\subsection{Example}

An extreme example of skew is caused by reverse indices (mapping
attribute values to their keys) on categorical data. Our worst-case
optimal dataflow join implementation relies heavily on reverse indices
on attributes (this is discussed in detail later in
\autoref{implementation}). Categorical data is the cornerstone of many
modern analytical tasks. Predictive models both consume and serve as
natural input sources of categorical inputs (\emph{features}) such as
\texttt{:transaction/fraudulent?} or \texttt{:person/bot?}.

While features can be modelled as boolean attributes, useful features
should be highly discriminating and thus apply to very small subsets
of entities with little overlap. Explicitely storing false values to
indicate absence of a feature is therefore a very storage-inefficient
approach. For this reason, a value of true is often meant to indicate
the presence of the feature on the given entity, while any entity not
contained in the relation is assumed to not posses the
feature. Modeling categorical features via binary relations in this
way leads to an extremely skewed distribution of values in the reverse
indices, which would then be stored entirely on a single worker.

\subsection{Remedies}

Skew in this scenario is caused by the requirement arranged state be
distributed according to its keys. To remediate this, we must modify
operator implementations to either still produce correct outputs when
individual workers might not see all inputs on a specific key or
obviate the need for the skewed indices entirely.

\cite{ammar2018distributed} propose a skew resilient version of the
\emph{Delta-BiGJoin} algorithm. Its implementation and adaption to 3DF
are beyond the scope of this work.

Another possibility is to distribute one side of a two-way join evenly
and broadcast all inputs on the other side to all workers. This is
advantageous in situations where the volume of inputs is low on one
side of the join. However, this approach does not rid us of the
sub-optimal modeling of features.

Instead, we can extend our worst-case optimal join framework slightly,
to cover unary relations. Unary relations bind only a single variable
and thus don't require reverse indices. \autoref{impl-hector} provides
a sketch of this extension. At the time of this writing, we have not
implemented it in 3DF.

\end{document}

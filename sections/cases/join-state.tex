\documentclass[../catalog.tex]{subfiles}

\begin{document}

\subsection{Problem}

Recall from \texttt{Data Model} our opinions on data modeling needs of
tomorrows information systems. In this world of normalized,
attribute-oriented data, relational joins cannot be treated as scary,
exceptional operations, but rather as commonplace, everyday tools.

Even the most basic aggregation of data, obtaining a consistent view
of domain entities over time, requires joining potentially many
attributes.

How to maintain robust, high-arity joins for many concurrent users
with low-latency and high-throughput is perhaps the defining problem
of the 3DF architecture.

As discussed in \texttt{@TODO}, two approaches for implementing joins of
arbitrary arity avail themselves to us. This section will focus on the
traditional one, where n-way joins are implemented as sequences of
two-way joins. Many stream processing systems materialize the output
stream of a two-way join into an indexed representation, if they are
to be-reused in other computations (rather than simply streamed
through to the client directly) \texttt{@TODO}. In Differential Dataflow
specifically, the `join` operator will arrange both of its input
collections.

This becomes problematic for 3DF, because any higher-arity join will
therefore maintain arranged result sets along the spine of the join
tree.

\subsection{Example}

Redundant join state becomes apparent in star-joins along highly
correlated columns, e.g. one-to-one relations with very similar
cardinalities and low selectivity. Unfortunately this is a very common
use case in our attribute-oriented data model. Consider the following
query.

``` clojure
[:find ?comment ?creationDate ?person ?ip ?browser \ldots{}
 :where
 [?comment :comment/creation-date ?creationDate]
 [?comment :comment/creator ?person]
 [?comment :comment/ip ?ip]
 [?comment :comment/browser ?browser]
 [?comment :comment/content ?content]
 [?comment :comment/place ?place]]
```

This query implies a six-way join on the comment ids, aggregating six
different comment attributes back into an aggregate view. Within the
LDBC dataset, every comment carries all six of these attributes,
s.t. all relations are of the same size, and map one-to-one onto each
other. Because of this, implementing this query as five consecutive
two-way joins would lead to roughly a threefold increase in the number
of tuples maintained by this dataflow.

[@TODO Joinjoinjoin plan illustration]

[@TODO result illustration]

Later on, we will encounter an even more pronounced version of this
problem, when looking at sub-graph queries, such as the triangle
search:

``` clojure
[:find ?a ?b ?c
 :where
 [?a :person/knows ?b]
 [?b :person/knows ?c]
 [?c :person/knows ?a]]
```

Here the attributes do not have a one-to-one correspondence
anymore. Rather any initial two-way join between two of the edges of
the triangle, will produce a result set proportional in size to the
sum of the squares of the degrees in the graph induced by the
`:person/knows` relation.

\subsection{Remedy}

[@TODO cite delta query sources]

Adapting techniques from the field of delta queries, we can break up
our queries along their potential sources of change. In our example,
any of the comment attributes may change arbitrarily, which means we
will end up with six separate queries.

[@TODO delta plan illustration]

What is different about the resulting plans, is that, given an indexed
representation of all attributes, they can be implemented in an
otherwise stateless manner. They do not need to maintain intermediate
arrangements as was the case for the traditional join operator.

[@TODO show results]

Assuming for a second that dataflow elements are essentially free
(they are not), this allows us to decouple the memory use of our
system from the number of active queries and the complexity of the
joins within, and instead bound it in the number and size of
attributes maintained.

There is a trade-off of course, as transforming plans into delta
queries leads to an increase in the number of dataflow elements. We
will see in our discussion of worst-case optimal streaming joins, that
this increase can even be quadratic in the number of relations
involved in the join.

Within any given delta flow, an attribute may only appear at a
position where either of its sides is already bound by the tuples
flowing through it, in order to avoid having to produce the entire
relation for each incoming tuple. We therefore would want to keep
attributes indexed in both directions (from entity id to value and the
other way around), to increase the number of possible plans available
to us.

Another subtle issue, identified in \texttt{0}, arises when implementing
delta queries within the dataflow model. Simultaneous updates to
multiple of the participating attributes can lead to redundant
derivations of the same result tuple. We must be careful to impose a
logical order on the execution.

[@TODO describe AltNeu]

\subsection{Sources}
% Emacs 25.3.1 (Org mode 8.2.10)
\end{document}

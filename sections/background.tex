\documentclass[../index.tex]{subfiles}

\begin{document}

3DF builds on a rather recent body of work that allows us to even
consider the continuous maintenance of queries, whose complexity used
to be the exclusive domain of analytical databases. In this chapter we
describe the dataflow model of computation as a foundation for
stream processing systems. We then provide intuition for the concept
of \emph{progress tracking}, and how it enables the efficient
execution of nested, cyclic dataflows. Finally, we discuss how
\emph{differential dataflow} extends the dataflow model to
\emph{incremental} operators and enables recursive queries through
independent, efficient iteration of regions of the dataflow.

\subsection{Dataflow}

\emph{Dataflow}, as commonly used, refers to two different but related
concepts, both of which are relevant to this work.

On the one hand, \emph{Dataflow} is an architectural paradigm in which
the execution of a computation is coordinated entirely via the
availability of data. This is in opposition to the von Neumann
architecture, which uses control structures (sequential statements,
branching via conditionals, loops) to coordinate the execution. Today,
we often describe systems that follow this idea as \emph{reactive}.

On the other hand, \emph{Dataflow programming} refers to a programming
paradigm, in which the structure of a computation is expressed as a
directed graph. This dataflow graph allows us to reason about
(distributed) execution of the computation. In particular, it allows
us to infer data dependencies and thus execution strategies that avoid
re-computation of unaffected paths through the graph on new inputs. In
the distributed setting, a fine-grained view of data dependencies
allows us to exploit more opportunities for concurrent execution.

\subsection{Progress Tracking}

\emph{Timely Dataflow} (\cite{timely}) is an implementation of the
cyclical dataflow model introduced in \cite{murray2013naiad}. It is
written in the Rust language. Timely Dataflow follows a data-parallel
approach, meaning each worker in a Timely cluster runs the same
dataflow computation with inputs partitioned between them.

Dataflow computations over unbounded, potentially out-of-order inputs
must reason about times at which it is safe to produce
results. Producing results over-eagerly will keep latencies low, but
causes downstream inconsistencies, because forwarded results do not
include all relevant inputs. Likewise, holding results back for too
long will ensure correctness but might increase latencies
unnecessarily (taken to the extreme, this end of the spectrum
corresponds to batch processing).

To solve this problem, a logical timestamp is attached to each datum
and an order is imposed on them. The ordering controls visibility,
thus detaching physical from logical availability. Taken together,
propagating information about input epochs along the dataflow graph
allows us to infer for which logical timestamps results might safely
be produced.

\emph{Progress tracking} is the problem of determining for each point
in the dataflow the set of input timestamps that may still be received
there. Timely Dataflow is a run-time for data-parallel dataflow
computations that coordinate via fine-grained progress tracking.

Without a co-ordination mechanism such as progress tracking, query
results will in general not reflect a consistent logical time. While
highly undesirable from a business point of view, this can also cause
non-terminating iterative computations.

\subsection{Differential Dataflow} \label{background-differential}

As the complexity of our computation grows and data volumes increase,
we will want to avoid re-computing results from scratch whenever an
input changes. Rather, we would like to incrementally update previous
results with (hopefully) small changes. In particular we would like to
do so even for iterative computations which arise in recursive
queries.

This model of computation is called \emph{incremental computation}. In
order to incrementalize a computation $f(X) = Y$ on some collection
$X$ we must find a corresponding $\delta{f}$ s.t. $f(X + \delta{x}) =
f(X) + \delta{f}(\delta{x}) = Y + \delta{y}$. To be of practical use,
we require $\delta{f}$ to only perform work proportional to the size
of $\delta{x}$. This can often be achieved whenever small input
changes cause small output changes, compared to the size of the entire
collection. In these cases we can hold on to the latest version of the
output collection $Y$ and use it to more efficiently compute
$\delta{y}$.

In a distributed setting with iterative computations, incremental
computation faces a significant challenge, because we must distinguish
the \emph{latest version} of a collection under two possible sources
of input (one from upstream and one from iterative
feedback). Assigning totally ordered timestamps effectively serializes
the computation, preventing us from computing iterations for batches
of inputs in parallel.

\emph{Differential computation} (\cite{mcsherry2013differential})
generalizes incremental computation by allowing timestamps to be
partially ordered. This allows us to represent multiple different
sources of inputs along different co-ordinates in a multi-dimensional
timestamp.

Differential Dataflow (\cite{differential}) is a programming framework
implementing differential computation on top of Timely. It allows to
express data-parallel dataflow programs as functional-relational
transformations over \emph{collections} of data, using incrementalized
implementations of well-known operators such as \texttt{map},
\texttt{filter}, \texttt{reduce}, and \texttt{join}. Any region of a
differential dataflow can be independently iterated to fixed
point. The entire computation responds efficiently to arbitrary
changes in its inputs. Collections implement multi-set semantics.

Finally, the dynamic environments we are targeting necessitate the
creation of new dataflows at runtime. New query dataflows will often
require access to inputs that pre-date their creation, in order to
give correct initial result sets. Having to retrieve historical inputs
from external input sources would incur not just significant
latencies, but also cause much tighter coupling than simple change
data capture.

Differential Dataflow has first-class support for compact in-memory
representation of historical collection traces, via so called
\emph{arrangements}. Arrangements maintain compacted, indexed batches
of updates to a collection. The resulting indexed state can be shared
between operators and allows for the dynamic creation of dataflows
that can feed off of them.

To our knowledge, Differential Dataflow is the only system that can
(a) maintain the performance characteristics of stream processors for
complex relational computations, such as required by recursive SQL
queries or the evaluation of Datalog rules, and that (b) allows for
low-latency, shared access to historical inputs.

\end{document}

\documentclass[../index.tex]{subfiles}

\begin{document}

3DF builds on a rather recent body of work that allows us to even
consider the continuous maintenance of queries, whose complexity used
to be the exclusive domain of analytical databases. In this chapter we
describe the dataflow model of computation as a foundation for
stream processing systems. We then provide intuition for the concept
of \emph{progress tracking}, and how it enables the efficient
execution of nested, cyclic dataflows. Finally, we discuss how
\emph{differential dataflow} extends the dataflow model to
\emph{incremental} operators and enables recursive queries through
independent, efficient iteration of regions of the dataflow.

\subsection{Dataflow}

\emph{Dataflow}, as commonly used, refers to two different but related
concepts, both of which are relevant to this work.

On the one hand, \emph{Dataflow} is an architectural paradigm in which
the execution of a computation is coordinated entirely via the
availability of data. This is in opposition to the von Neumann
architecture, which uses control structures (sequential statements,
branching via conditionals, loops) to coordinate the execution. Today
we often describe systems that follow this idea as \emph{reactive}.

On the other hand, \emph{Dataflow programming} refers to a programming
paradigm in which the structure of a computation is expressed as a
directed graph. This dataflow graph allows us to reason about the
(potentially distributed) execution of the computation. In particular,
it allows us to infer data dependencies and thus execution strategies
that avoid recomputation of unaffected paths through the graph on new
inputs. In the distributed setting, a fine-grained view of data
dependencies allows us to exploit more opportunities for concurrent
execution.

\subsection{Progress Tracking}

\emph{Timely Dataflow} (\cite{timely}) is an implementation of the
cyclical dataflow model introduced in \cite{murray2013naiad}. It is
written in the Rust language. Timely Dataflow follows a data-parallel
approach, meaning each worker in a Timely cluster runs the exact same
dataflow computation, with inputs partitioned between them.

Dataflow computations over unbounded, potentially out-of-order inputs
must reason about times at which it is safe to produce
results. Producing results over-eagerly will keep latencies low, but
causes downstream inconsistencies, due to forwarded results not
incorporating all relevant inputs. Likewise, holding results back for
too long ensures correctness, but increases latencies
unnecessarily. Taken to the extreme, this end of the spectrum
corresponds to batch processing.

To solve this problem, a logical timestamp is attached to each datum
and an order is imposed on them. The ordering controls visibility,
thus detaching physical from logical availability. Taken together,
propagating information about input epochs along the dataflow graph
allows us to infer the set of logical timestamps for which it is safe
to produce results.

\emph{Progress tracking} is the problem of determining for each point
in the dataflow the set of input timestamps that may still be received
there. This set is represented by a \emph{frontier} consisting of one
or more mutually incomparable times. Inputs might still be received
for any time in advance of any of the frontier elements. Conversely,
results for times earlier than the frontier can be propagated
safely. Timely Dataflow is a runtime for data-parallel dataflow
computations that coordinate via fine-grained progress tracking.

Without a coordination mechanism like progress tracking, query results
will in general not reflect a consistent logical time. While highly
undesirable from a business point of view, this can also cause
non-terminating iterative computations.

\subsection{Differential Dataflow} \label{background-differential}

As the complexity of our computation grows and data volumes increase,
we will want to avoid recomputing results from scratch whenever an
input changes. Rather, we would like to incrementally update previous
results with (hopefully) small changes. In particular, we would like
to do so even for iterative computations, which arise in the
evaluation of recursive queries.

This model of computation is called \emph{incremental computation}. In
order to incrementalize a computation $f(X) = Y$ on some collection
$X$ we must find a corresponding $\delta{f}$ s.t. $f(X + \delta{x}) =
f(X) + \delta{f}(\delta{x}) = Y + \delta{y}$. To be of practical use,
we require $\delta{f}$ to only perform work proportional to the size
of $\delta{x}$. This can often be achieved whenever small input
changes cause small output changes, as compared to the size of the
entire collection. In these cases we can hold on to the latest version
of the output collection $Y$ and use it to more efficiently compute
$\delta{y}$.

In a distributed setting with iterative computations, incremental
computation faces a significant challenge because we must distinguish
the \emph{latest version} of a collection under two possible sources
of input (one from upstream and one from iterative
feedback). Assigning totally ordered timestamps prevents us from
computing iterations for batches of inputs in parallel, effectively
serializing the computation.

\emph{Differential computation} (\cite{mcsherry2013differential})
generalizes incremental computation by allowing timestamps to be
partially ordered. This allows us to represent multiple different
sources of inputs along different coordinates in a multi-dimensional
timestamp.

\emph{Differential Dataflow} (\cite{differential}) is a programming
framework implementing differential computation on top of Timely. It
allows for the expression of data-parallel dataflow programs as
functional-relational transformations over \emph{collections} of data,
using incrementalized implementations of well-known operators such as
\texttt{map}, \texttt{filter}, \texttt{reduce}, and \texttt{join}. Any
region of a differential dataflow can be independently iterated to
fixed point. The entire computation responds efficiently to arbitrary
changes in its inputs. Collections implement multiset semantics.

Finally, the dynamic environments we are targeting necessitate the
creation of new dataflows at runtime. New query dataflows will often
require access to inputs that predate their creation, in order to give
correct initial result sets. Having to retrieve historical inputs from
external input sources would incur not just significant latencies, but
also cause much tighter coupling than simple change data capture.

Differential Dataflow has first-class support for compact in-memory
representations of historical collection traces, via so called
\emph{arrangements}. Arrangements maintain compacted, indexed batches
of updates to a collection. The resulting indexed state can be shared
between operators, and allows for the efficient, dynamic creation of
dataflows that can feed off of them.

To our knowledge, Differential Dataflow is the only system that can
(a) maintain the performance characteristics of stream processors for
complex relational computations, as required by recursive SQL queries
or the evaluation of Datalog rules, and that (b) allows for
low-latency, shared access to historical inputs.

\end{document}

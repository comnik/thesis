\documentclass{article}
\usepackage[utf8]{inputenc}
\usepackage[T1]{fontenc}
\usepackage{fixltx2e}
\usepackage{graphicx}
\usepackage{subcaption}
\usepackage{longtable}
\usepackage{float}
\usepackage{wrapfig}
\usepackage{rotating}
\usepackage[normalem]{ulem}
\usepackage{amsmath}
\usepackage{textcomp}
\usepackage{marvosym}
\usepackage{wasysym}
\usepackage{amssymb}
\usepackage{hyperref}
\usepackage{verbatim}
\usepackage{natbib}
\usepackage[masterthesis]{systems-cover}
\usepackage{subfiles}
\usepackage{listings, listings-rust, listings-clojure, listings-json, listings-datalog}

\covernum{ 263-0800-00L	}
\covertitle{ Optimising Distributed Dataflows in Interactive Environments }
\coverauthor{ Nikolas Göbel }
\coversupervisedby{ Frank McSherry, Gustavo Alonso }
\coverdate{ November 2018 - April 2019 }

\begin{document}

\tableofcontents

\newpage

\section{Abstract} \label{abstract}

The capability to maintain sophisticated relational queries over
potentially unbounded data streams is a critical competitive advantage
for data-driven organizations. Traditional approaches to relational
query optimization assume ad-hoc execution, benevolent distribution of
data, and the availability of accurate statistics at planning
time. Established practice in low-latency stream processing assumes
comparatively simpler computational models and relaxed consistency
guarantees.

\emph{Differential computation} generalizes incremental computation to
support data-parallel relational operators and arbitrarily nested,
least fixed point iteration. In this work we identify the challenges
to providing these new capabilities to dynamic multi-user environments
and describe solutions adapted from the database and stream processing
communities.

We implement our findings in \emph{3DF}, an interactive Datalog engine
built on \emph{Differential Dataflow}, an implementation of
differential computation. We evaluate the resulting system and
showcase in particular (I) the use of delta-queries to maintain joins
of arbitrary arity with a constant memory footprint, (II) the use of
worst-case optimal join algorithms to provide predictable query
performance on adversarially skewed data, and (III) strategies to
share resources between clients without giving up individual
optimization opportunities.

\newpage

\section{Introduction and Motivation} \label{intro}
\subfile{sections/intro.tex}
\newpage

\section{Problem Description and Scope} \label{problem}
\subfile{sections/problem.tex}
\newpage

\section{Background} \label{background}
\subfile{sections/background.tex}
\newpage

\section{Design and Implementation of 3DF} \label{3df}
\subfile{sections/3df.tex}
\newpage

\section{Known Techniques and Related Work} \label{known-techniques}
\subfile{sections/known-techniques.tex}
\newpage

\subfile{sections/catalog.tex}

\section{Implementation Details} \label{implementation}
\subfile{sections/implementation.tex}

\section{Other Findings and Future Work} \label{future-work}
\subfile{sections/future-work.tex}
\newpage

\section{Conclusions} \label{conclusions}

We have investigated the problem of optimizing the continuous
maintenance of relational queries for dynamic, multi-user
environments. Besides the basic need for consistent, correct results,
we identified robustness, support for many concurrent users with
general-purpose use cases, and low-latency execution at
high-throughputs as the driving goals behind this work. We thus were
looking for techniques that could bring the domain of analytical, and
graph databases closer to that of high-performance stream processing.

In closing, we want to summarize our findings and how they lead to a
system that satisfies these properties.

\begin{enumerate}
\item
  The cyclic dataflow model in combination with progress tracking of
  partially-ordered logical times, as implemented by the Timely
  Dataflow framework, enables consistent, data-parallel computations
  that exploit fine-grained opportunities for concurrent
  execution. Support for local iteration of regions of the dataflow
  dramatically increase the class of algorithms that can be expressed
  within this model.

\item
  Distributed incremental computation with iteration and support for
  retractions, as implemented by the Differential Dataflow framework,
  makes it possible to efficiently maintain complex relational views
  over data streams.

\item
  Compact, shared representations of historical inputs, as implemented
  by Differential's arrangements, allow the dynamic creation of new
  dataflows with low-latency.

\item
  Declarative, relational languages such as Datalog, in combination
  with an attribute-oriented data model, are expressive enough to
  support a wide range of use cases, and are easily extended to new
  domains, without fundamentally affecting the underlying runtime.

\item
  Deferred synthesis of query plans is essential in dynamic
  environments, where queries might be defined across multiple
  declarative interactions, and where partially constrained plans
  significantly impact the overall system performance.

\item
  Exploiting the linearity of the relational join operator via delta
  queries makes it possible to implement joins of arbitrary arity with
  a constant memory footprint and without significantly impacting
  latencies. Delta queries can be combined to form join operators that
  update incrementally in response to arbitrary input changes.

\item
  Worst-case optimal dataflow join algorithms are compatible with
  delta queries and enable robust, predictable performance
  irrespective of clause order. The performance of the resulting
  operators can be competitive with traditional join-at-a-time
  approaches for best- and average-case queries, and offers asymptotic
  improvements for a class of relevant queries. Crucially, these
  properties are preserved on long-running dataflows, even as the
  underlying data distributions change, without the need for
  synchronized re-optimization.

\item
  Extensions beyond purely conjunctive queries can be implemented
  within the same worst-case optimal join framework. We gave
  implementations of binary predicates, stratified negation, and
  constant bindings. Combined with Differential's support for
  fixed-point iteration, these cover the needs of many
  functional-relational query languages. Similarly, we gave an
  extension to unary relations, that allows remediating categorical
  data as a common source of skew.

\item
  Modeling client parameters via additional joins on parameter
  relations allows both the dynamic re-configuration of existing
  dataflows, as well as the multiplexing of multiple client's
  interests onto a single dataflow. The resources of that dataflow can
  then be shared without limiting optimization opportunities. More
  generally applicable re-use is still possible via arrangements and
  scales significantly better than creating individual dataflows for
  every client.

\end{enumerate}

We identified areas for future work, in particular on automated
policies and hybrid planners, that are required to tie these
primitives into fully-formed systems. Nevertheless we believe that our
findings and their implementation within 3DF provide strong
foundations on which such systems can build.

\newpage

\bibliographystyle{te}
\bibliography{references}

\end{document}

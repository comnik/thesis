\documentclass{article}
\usepackage[utf8]{inputenc}
\usepackage[T1]{fontenc}
\usepackage{fixltx2e}
\usepackage{graphicx}
\usepackage{subcaption}
\usepackage{longtable}
\usepackage{float}
\usepackage{wrapfig}
\usepackage{rotating}
\usepackage[normalem]{ulem}
\usepackage{amsmath}
\usepackage{textcomp}
\usepackage{marvosym}
\usepackage{wasysym}
\usepackage{amssymb}
\usepackage{hyperref}
\usepackage{verbatim}
\usepackage{natbib}
\usepackage[masterthesis]{systems-cover}
\usepackage{subfiles}
\usepackage{listings, listings-rust, listings-clojure, listings-json}

\covernum{ 263-0800-00L	}
\covertitle{ Optimising Distributed Dataflows in Interactive Environments }
\coverauthor{ Nikolas Göbel }
\coversupervisedby{ Frank McSherry, Gustavo Alonso }
\coverdate{ November 2018 - April 2019 }

\begin{document}

\tableofcontents

\newpage

\section{Abstract} \label{abstract}

The capability to maintain sophisticated relational queries over
potentially unbounded data streams is a critical competitive advantage
for data-driven organizations. Traditional approaches to relational
query optimization assume ad-hoc execution, benevolent distribution of
data, and the availability of accurate statistics at planning
time. Established practice in low-latency stream processing assumes
comparatively simpler computational models and relaxed consistency
guarantees.

\emph{Differential computation} generalizes incremental computation to
support data-parallel relational operators and arbitrarily nested,
least fixed point iteration. In this work we identify the challenges
to providing these new capabilities to dynamic multi-user environments
and describe solutions adapted from the database and stream-processing
communities.

We implement our findings in \emph{3DF}, an interactive Datalog engine
built on \emph{Differential Dataflow}, an implementation of
differential computation. We evaluate the resulting system and
showcase in particular (I) the use of delta-queries to maintain joins
of arbitrary arity with a constant memory footprint, (II) the use of
worst-case optimal join algorithms to provide predictable query
performance on adversarially skewed data, and (III) strategies to
share resources between clients without giving up individual
optimization opportunities.

\newpage

\section{Introduction and Motivation} \label{intro}
\subfile{sections/intro.tex}
\newpage

\section{Problem Description and Scope} \label{problem}
\subfile{sections/problem.tex}
\newpage

\section{Background} \label{background}
\subfile{sections/background.tex}
\newpage

\section{Design and Implementation of 3DF} \label{3df}
\subfile{sections/3df.tex}
\newpage

\section{Known Techniques and Related Work} \label{known-techniques}
\subfile{sections/known-techniques.tex}

\subfile{sections/catalog.tex}

\section{Implementation Details} \label{implementation}
\subfile{sections/implementation.tex}

\section{Other Findings and Future Work} \label{future-work}
\subfile{sections/future-work.tex}

\section{Conclusions} \label{conclusions}

\newpage

\bibliographystyle{te}
\bibliography{references}

\end{document}
